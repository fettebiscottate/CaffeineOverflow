\title{CaffeineOverflow, Piano di Sviluppo}
\documentclass[paper=a4, fontsize=11pt]{scrartcl}
\usepackage[T1]{fontenc}
\usepackage{fourier}
\usepackage[colorlinks=true, urlcolor=blue]{hyperref}
\usepackage[italian]{babel}														
\usepackage[protrusion=true,expansion=true]{microtype}	
\usepackage{amsmath,amsfonts,amsthm} 
\usepackage[pdftex]{graphicx}	
\usepackage{url}
\usepackage[utf8x]{inputenc}
\usepackage{booktabs}
\usepackage{siunitx}
%%% Custom sectioning
\usepackage{sectsty}
\allsectionsfont{\centering \normalfont\scshape}
\usepackage{float}
\renewcommand{\arraystretch}{1.1}

%%% Custom headers/footers (fancyhdr package)
\usepackage{fancyhdr}
\pagestyle{fancyplain}
\fancyhead{}											% No page header
\fancyfoot[L]{}											% Empty 
\fancyfoot[C]{}											% Empty
\fancyfoot[R]{\thepage}									% Pagenumbering
\renewcommand{\headrulewidth}{0pt}			% Remove header underlines
\renewcommand{\footrulewidth}{0pt}				% Remove footer underlines
\setlength{\headheight}{13.6pt}


%%% Equation and float numbering
\numberwithin{equation}{section}		% Equationnumbering: section.eq#
\numberwithin{figure}{section}			% Figurenumbering: section.fig#
\numberwithin{table}{section}				% Tablenumbering: section.tab#


%%% Maketitle metadata
\newcommand{\horrule}[1]{\rule{\linewidth}{#1}} 	% Horizontal rule

\title{
		%\vspace{-1in} 	
		\usefont{OT1}{bch}{b}{n}
		\normalfont \normalsize \textsc{ALMA MATER STUDIORUM- Università di Bologna} \\ [25pt]
		\horrule{0.5pt} \\[0.4cm]
		\huge CaffeineOverflow, Piano di Sviluppo \\
		\horrule{2pt} \\[0.5cm]
}
\author{ Gruppo "CaffeineOverflow"\\
		\normalfont 								\normalsize
        Giacomo Minello, Matteo Tramontano, Davide Menetto\\[-3pt]		\normalsize
        \today
}
\date{}
\usepackage{xcolor}
\usepackage{minted}
\definecolor{LightGray}{gray}{0.9}
\floatstyle{boxed} 
\restylefloat{figure}
%%% Begin document
\begin{document}
\maketitle
\section{Introduzione}
\subsection{Overview del Progetto}
Il progetto consiste nella creazione di un di un sito web. Lo scopo di questo sito è permettere agli utenti iscritti di porre domande e fornire risposte agli interrogativi proposti.
\subsection{Artefatti del Progetto}
\begin{itemize}
\item Diario di progetto
\item Modello dei casi d'uso
\item Modello di dominio
\item Glossario
\item Analisi dei rischi
\item Descrizione dell'architettura
\item Manuale utente
\item Documentazione
\end{itemize}
\subsection{Materiali di riferimento}
In questa sezione elenchiamo i documenti a cui faremo riferimento nel piano di progetto:
\begin{itemize}
    \item Slides del corso di Ingegneria del Software
\end{itemize}
\subsection{Definizioni e Abbreviazioni}
Il glossario definisce i termini appertenenti al dominio del progetto CaffeineOverflow. Una copia del glossiario è disponibile tra gli allegati del progetto.
\section{Organizzazione del Progetto}
\subsection{Modello del Processo}
Per la realizzazione della nostra applicazione software riteniamo opportuno utilizzare il modello consigliato ovvero un ibrido strutturato-agile composto da due fasi: inception e construction. Riteniamo inoltre utile aggiungere una fase intermedia di Elaboration al fine di gestire in modo ottimale i rischi del progetto.\newline
Il modello risultante prevede nella fase di inception uno studio preciso sulla fattibilità della realizzazione del progetto, nella fase di Elaboration una validazione dell'architettura e l'implementazione di un'architettura di base e nella fase di construction l'implementazione delle funzioni del sistema secondo le metodologie Aglie di Scrum E XP. \newline
Il numero di release previste è pari al numero di iterazioni della fase di Construction. Sono previste tre Milestones, di cui l'ultima coincide con la realizzazione del progetto.
GLi issue del progetto sono stati tracciati nell'issue traker fornto da Bitbucket ed organizzati per tipologia.
\subsection{Struttura Organizzativa}
(Srum master, product owner a rotazione)
Come indicato nella consegna del progetto due componenti del gruppo giocano il doppio ruolo di membri del team e di scrum master e product owner (i ruoli si ridistribuiscono al termine di ogni sprint). 
\section{Descrizione dei Processi Gestionali}
\subsection{Obiettivi e Priorità}
L’obiettivo consiste nella creazione di una piattaforma funzionante che permetta di inserire domande e risposte. 
Le nostre priorità sono: rispettare puntualmente i tempi di consegna stabiliti svolgendo in modo efficiente ed efficace i compiti assegnati, mantenere un buon grado di collaborazione tra i componenti del gruppo e garantire un prodotto finale ben strutturato ed affidabile.
\subsection{Gestione dei rischi}
Il progetto è soggetto a possibili rischi che, se non affrontati nella maniera adeguata, potrebbero alterare la
buona riuscita della realizzazione del progetto, portando così ad un fallimento. Un'analisi dei rischi è disponibile tra gli allegati del progetto.
\section{Pianificazione del lavoro}
\subsection{WBS (Work breakdown structure)}
\begin{enumerate}
 \item Specifiche
  \begin{itemize}
      \item Definizione obiettivi
      \item Definizione vincoli
      \item Definizione modello del progetto
      \item Definizione e gestione rischi
      \item Stesura piano di progetto
    \end{itemize}
\item Progettazione concettuale
    \begin{itemize}
      \item Modellazione del sistema
      \item Definizione dei requisiti
      \item Stesura documento di analisi e specifica
    \end{itemize}

\item  Progettazione tecnica
\begin{itemize}
      \item Stesura documento di progettazione
    \end{itemize}
\item Client
\begin{itemize}
      \item Creazione User-Interface
    \end{itemize}
\item Testing
\item Rilascio Software 
\end{enumerate}
\subsection{Pianificazione}
\begin{itemize}
  \item Inizio: 26 Giugno 2019
  \item Completamento Inception: 
  \item Completamento Elaboration: 
  \item Completamento prima Milestone: 
  \item Completamento seconda Milestone: 
  \item Completamento terza Milestone: 
  \item Completamneto Testing finale: 
  \item Consegna Software: 6 Luglio 2019.
\end{itemize}
\\
\begin{flushright}
Giacomo Minello
\end{flushright}
\\~\\
\begin{flushright}
Matteo Tramontano
\end{flushright}
\\~\\
\begin{flushright}
Davide Menetto
\end{flushright}
%%% End document
\end{document}