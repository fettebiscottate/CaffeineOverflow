\title{CaffeineOverflow, Glossario}
\documentclass[paper=a4, fontsize=11pt]{scrartcl}
\usepackage[T1]{fontenc}
\usepackage{fourier}
\usepackage[colorlinks=true, urlcolor=blue]{hyperref}
\usepackage[italian]{babel}														
\usepackage[protrusion=true,expansion=true]{microtype}	
\usepackage{amsmath,amsfonts,amsthm} 
\usepackage[pdftex]{graphicx}	
\usepackage{url}
\usepackage[utf8x]{inputenc}
\usepackage{booktabs}
\usepackage{siunitx}
%%% Custom sectioning
\usepackage{sectsty}
\allsectionsfont{\centering \normalfont\scshape}
\usepackage{float}
\renewcommand{\arraystretch}{1}

%%% Custom headers/footers (fancyhdr package)
\usepackage{fancyhdr}
\pagestyle{fancyplain}
\fancyhead{}											% No page header
\fancyfoot[L]{}											% Empty 
\fancyfoot[C]{}											% Empty
\fancyfoot[R]{\thepage}									% Pagenumbering
\renewcommand{\headrulewidth}{0pt}			% Remove header underlines
\renewcommand{\footrulewidth}{0pt}				% Remove footer underlines
\setlength{\headheight}{13.6pt}


%%% Equation and float numbering
\numberwithin{equation}{section}		% Equationnumbering: section.eq#
\numberwithin{figure}{section}			% Figurenumbering: section.fig#
\numberwithin{table}{section}				% Tablenumbering: section.tab#


%%% Maketitle metadata
\newcommand{\horrule}[1]{\rule{\linewidth}{#1}} 	% Horizontal rule

\title{
		%\vspace{-1in} 	
		\usefont{OT1}{bch}{b}{n}
		\normalfont \normalsize \textsc{ALMA MATER STUDIORUM- Universit\'a di Bologna} \\ [25pt]
		\horrule{0.5pt} \\[0.4cm]
		\huge CaffeineOverflow, Glossario \\
		\horrule{2pt} \\[0.5cm]
}
\author{ Gruppo "CaffeineOverflow"\\
		\normalfont 								\normalsize
        Giacomo Minello, Matteo Tramontano, Davide Menetto\\[-3pt]		\normalsize
        \today
}
\date{}
\usepackage{xcolor}
\usepackage{minted}
\definecolor{LightGray}{gray}{0.9}
\floatstyle{boxed} 
\restylefloat{figure}
%%% Begin document
\begin{document}
\maketitle
\section{Introduzione}
\paragraph{IL glossario} definisce i termini appertenenti al dominio del progetto CaffeineOverflow. \newline
Le voci non sono ordinate. Le voci possono consistere in una parola, una frase o un acronimo. Alcuni esempi e note sono state aggiunte per rendere più chiare alcune definizioni.
Alcune termini rererenziano altri termini o includono sinonimi. 


\section{Glossario}

	\begin{table}[H]
    \centering
    \begin{tabular}{c p{9cm} p{4cm} } \toprule
    {$Termine$}  & {$Descrizione$} & {$Sinonimi$} \\ \midrule
    Visitatore & Si tratta di un utente non registrato, ha a disposizione funzionalità limitate rispetto agli utenti registrati, può visualizzare solo le ultime domande e risposte,
    effettuate da altri utenti, senza poter effettuare alcun inserimento. & Utente non registrato\\
    Utente & Si intende l'utente registrato senza "poteri" aggiuntivi, può inserire domande e rispondere alle domande effettuate da altri utenti. & Utente registato, Utente semplice\\
    Giudice & Utente registrato, con funzionalità aggiuntive rispetto               
                                    agli altri utenti, nominato dall'admin, ha il potere di valutare
                                    le risposte degli altri utenti e di elimare le risposte che 
                                    ritiene scorrette o inadeguate (specializzazione dell'utente
                                    semolice).
    & Utente premium\\
    Admin & Specializzazione dell'utente giudice, oltre alle funzionalità        
                                    dell'utente semplice e del giudice (che non ha interesse ad 
                                    utilizzare) ha il potere di eliminare le domande, di eleggere
                                    i giudici, aggiungere le categorie nonchè modicare quelle 
                                    esistenti. E' l'amministratore e il cordinatore della piattaforma. & Amministratore, Cordinatore, Regolatore\\
    Domanda & Posta da un utente registrato è composta da un testo di massimo       
                                    300 caratteri, opzionale collegamento a immagini e data ed ora 
                                    di pubblicazione.   & Quesito\\
    Risposta & Effettuata da un qualsiasi utente registrato è composta da massimo    
                                    500 caratteri, opzionale collegamento a immagini e data ed ora 
                                    di pubblicazione. & \\
    Registrazione & Consiste nell' inserimento obbligatorio di username, password ed       
                                    email (più altri dati opzionali quali nome, cognome, sesso, data
                                    di nascita, luogo di nascita, residenza e social network), salva 
                                    dati nel sistema e comporta il passaggio da visitatore a utente
                                    semplice. & Sign up\\
    Autenticazione & Inserimento di email e password al fine di essere riconosciuti dal     
                                    sistema, necessaria per tutte le categorie, a ovvia esclusione del
                                    visitatore.    & Login\\
    Valutazione & Viene assegnato dai giudici alle risposte di un qualsiasi utente       
                                    (registrato), consiste in un voto che può assumere valori compresi
                                    tra lo 0 e il 5.   & Voto, Giudizio\\
    Categoria & Si riferiscono alle domande (e alle relative risposte) indicandone    
                                    l'argomento principale. Possono essere create o modificate unicamente
                                    dall'admin del sistema e possono avere sotto categorie a cascata.         
               & Argomento\\
     \bottomrule
     \end{tabular}
    \end{table}
 
    

\\
\begin{flushright}
Giacomo Minello
\end{flushright}
\\~\\
\begin{flushright}
Matteo Tramontano
\end{flushright}
\\~\\
\begin{flushright}
Davide Menetto
\end{flushright}
%%% End document
\end{document}