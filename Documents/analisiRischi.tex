\title{CaffeineOverflow, Analisi dei Rischi}
\documentclass[paper=a4, fontsize=11pt]{scrartcl}
\usepackage[T1]{fontenc}
\usepackage{fourier}
\usepackage[colorlinks=true, urlcolor=blue]{hyperref}
\usepackage[italian]{babel}														
\usepackage[protrusion=true,expansion=true]{microtype}	
\usepackage{amsmath,amsfonts,amsthm} 
\usepackage[pdftex]{graphicx}	
\usepackage{url}
\usepackage[utf8x]{inputenc}
\usepackage{booktabs}
\usepackage{siunitx}
%%% Custom sectioning
\usepackage{sectsty}
\allsectionsfont{\centering \normalfont\scshape}
\usepackage{float}
\renewcommand{\arraystretch}{1.1}

%%% Custom headers/footers (fancyhdr package)
\usepackage{fancyhdr}
\pagestyle{fancyplain}
\fancyhead{}											% No page header
\fancyfoot[L]{}											% Empty 
\fancyfoot[C]{}											% Empty
\fancyfoot[R]{\thepage}									% Pagenumbering
\renewcommand{\headrulewidth}{0pt}			% Remove header underlines
\renewcommand{\footrulewidth}{0pt}				% Remove footer underlines
\setlength{\headheight}{13.6pt}


%%% Equation and float numbering
\numberwithin{equation}{section}		% Equationnumbering: section.eq#
\numberwithin{figure}{section}			% Figurenumbering: section.fig#
\numberwithin{table}{section}				% Tablenumbering: section.tab#


%%% Maketitle metadata
\newcommand{\horrule}[1]{\rule{\linewidth}{#1}} 	% Horizontal rule

\title{
		%\vspace{-1in} 	
		\usefont{OT1}{bch}{b}{n}
		\normalfont \normalsize \textsc{ALMA MATER STUDIORUM- Universit\'a di Bologna} \\ [25pt]
		\horrule{0.5pt} \\[0.4cm]
		\huge CaffeineOverflow, Analisi dei Rischi \\
		\horrule{2pt} \\[0.5cm]
}
\author{ Gruppo "CaffeineOverflow"\\
		\normalfont 								\normalsize
        Giacomo Minello, Matteo Tramontano, Davide Menetto\\[-3pt]		\normalsize
        \today
}
\date{}
\usepackage{xcolor}
\usepackage{minted}
\definecolor{LightGray}{gray}{0.9}
\floatstyle{boxed} 
\restylefloat{figure}
%%% Begin document
\begin{document}
\maketitle
\section{Introduzione}
\paragraph{Il documento} serve a descrive i rischi
\section{Rischi}
Il progetto è soggetto a possibili rischi che, se non affrontati nella maniera adeguata, potrebbero alterare la
buona riuscita della realizzazione del progetto, portando così ad un fallimento.
I possibili rischi sono:
\begin{itemize}
\item Malfunzionamento dei sistemi informatici: è possibile che si presentino dei problemi relativi al
funzionamento non corretto dei sistemi utilizzati durante la fase di realizzazione del progetto. Una
possibile soluzione ad esempio, nel caso in cui un computer personale non funzioni, ricorrere alle
macchine presenti nei laboratori dell’università.
\item Difficoltà di utilizzo degli strumenti: può capitare che i componenti del gruppo riscontrino
difficoltà nell’impiego dei mezzi disponibili. E’ necessario in tal caso, colmare queste lacune
conoscitive mediante lo studio e l’approfondimento dei concetti mancanti.
\item Difficoltà nello sviluppo delle componenti di progetto: è possibile che a causa di un calcolo
errato dei tempi o della fattibilità di certe componenti, si abbiano delle difficoltà nello sviluppo di
alcune parti del progetto. E’ necessario, in tale evenienza, ricercare le soluzioni anche ricorrendo
all’aiuto degli altri membri del gruppo.
\item Perdita dei dati: non è da escludere la possibilità di perdita di materiale. Per questa ragione, è bene
prevenire un simile imprevisto provvedendo a backup regolari dei dati prodotti nel corso della
realizzazione del progetto.
\end{itemize}
Di seguito riportiamo la tabella dei rischi.

	\begin{table}[H]
    \centering
    \begin{tabular}{p{5cm} c c p{5cm} } \toprule
    {$Rischio$}  & {$Probabilità$} & {$Impatto$} & {$Azione$}\\ \midrule
    Malfunzionamento dei sistemi informatici & Molto Bassa & Basso & Sostituzione strumento \\
    Difficoltà di utilizzo degli strumenti & Bassa & Basso & Studio personale \\
    Difficoltà nello sviluppo dei componenti di progetto & Media & Medio & Ricerca di soluzioni adeguate\\
    Perdita dei dati & Bassa & Molto alto & Utilizzo di sistemi di backup e versionamento \\
     \bottomrule
     \end{tabular}
    \end{table}

\section{Meccanismi di monitoraggio e controllo}
Durante ogni fase del progetto verranno svolte delle attività di controllo in cui ogni membro del gruppo potrà
esaminare la correttezza e la qualità del lavoro svolto dagli altri secondo le metodologie Scrum. Si tratterà inoltre di momenti per
scambiare nuove idee, proposte di miglioramento, spiegare le difficoltà incontrate e proporre soluzioni ai
problemi. 


\\
\begin{flushright}
Giacomo Minello
\end{flushright}
\\~\\
\begin{flushright}
Matteo Tramontano
\end{flushright}
\\~\\
\begin{flushright}
Davide Menetto
\end{flushright}
%%% End document
\end{document}